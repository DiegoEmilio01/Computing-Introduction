\documentclass[12pt]{article}
\usepackage[utf8]{inputenc}
\usepackage{amsmath}
\usepackage{amssymb}
\usepackage{multicol}
\usepackage{fullpage}
\usepackage{bera}
\usepackage{xcolor}
\usepackage{hyperref}

\setlength{\parindent}{0pt}

\begin{document}

\begin{flushleft}
{\footnotesize Pontificia Universidad Católica de Chile\\
Departamento de Ciencia de la Computación\\
Computación: Ciencia y Tecnología del Mundo Digital\\
}
\begin{center}
{\huge\bf Tarea Chica 3: Máquinas de Turing}\\ \vspace{0.5cm}
Profesor Denis Parra \\

\rule{\linewidth}{0.1mm}
\end{center}
\end{flushleft}

% El informe no debe contener esta sección
\section*{Comandos básicos \LaTeX}

Para escribir en modo matemático entre medio del texto usamos los signos pesos $f(a) = 5 * a - 10 + 3$ y todo lo que quede dentro será interpretado como modo matemático. El doble slash sirve como salto de línea. \\

Para escribir una línea centrada de modo matemático usamos doble signo peso:
$$f(a) = 5 * a - 10 + 3$$

Cuando estamos en modo matemático algunos caracteres especiales (como los corchetes) deben escaparse con slash. Por lo tanto, si queremos definir un conjunto lo haríamos del siguiente modo $A = \{a, b, c, d, e, B\}$\\

Para usar subindices usamos el guión bajo seguido de corchetes que encierran el índice $q_{0}, q_{1}, q_{f}, q_{algo}$. Para usar super índices usamos el acento cincunflejo $q^{0}, q^{f}, q^{algo}$. Incluso pueden combinarse $q_{0}^{1}$.\\

Además de los caracteres clásicos también podemos usar letras griegas en minúsculas $\alpha, \beta, \gamma, \delta, \sigma$ o mayúsculas $\Pi, \Gamma, \Delta, \Sigma$. También hay flechas $\leftarrow, \rightarrow, \leftrightarrow$ y todo tipo de símbolos $\--, \in, \forall, \exists, \wedge, \vee, \cup, \cap$ siempre usando el slash para definirlos. Es difícil memorizar los comandos para todos los símbolos, pero en la página \url{http://detexify.kirelabs.org/classify.html} puedes dibujarlos y te dirá el comando. \\

Para cosas más específicas se puede buscar en Google, hay mucha información. Las operaciones matemáticas se ven muy bien $\frac{2}{3}, \sqrt{2}, \log_{2}(10), 2^{3^{4}} \neq (2^{3})^{4}, \{1, 2, 3\} \times \{a, b, c\}, 2 \cdot 3 \div 4$\\

También se pueden usar contextos para definir \textit{cursiva}, \textbf{negrita}, \texttt{letra de máquina}, etc. En modo matemático no se ve bien el texto porque no hay espacios $L = \{ w | w es multiplo de tres\}$ por lo que resulta útil usar el contexto de texto dentro del modo matemático $L = \{ w | w \text{ es multiplo de tres}\}$.

\subsection*{Ordenar la información}
Para mostrar la información podemos utilizar:
\begin{itemize}
		        \item Listas:
		        \begin{enumerate}
		            \item Enumeradas
		            \item Desordenadas
		        \end{enumerate}
		        \item Tablas
	        \end{itemize}

Por ejemplo, esta es una tabla con bordes visibles:
		    \[
		    \begin{array}{|c|c|}
		    \hline
		         \text{Símbolo/Comando} & \text{Que hace}  \\ \hline
		         \text{|} & \text{Indica que hay una línea vertical} \\
		         \text{\&} & \text{Separa la información entre columnas} \\
		         \text{hline} & \text{Coloca una línea horizontal} \\
		         \hline
		    \end{array}
		    \]
\newpage





\end{document}
